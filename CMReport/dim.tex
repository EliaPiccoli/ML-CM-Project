\documentclass{article}
\usepackage{graphicx} % for inserting images
\usepackage[utf8]{inputenc} % for good practice
\usepackage{amsmath} % for equations
\usepackage{amsfonts}
\usepackage{hyperref}
\usepackage[ruled,linesnumbered]{algorithm2e}
\usepackage{setspace}
\usepackage{siunitx}
\sisetup{output-exponent-marker=\ensuremath{\mathrm{e}}}

\newcommand{\norm}[1]{\left\lVert#1\right\rVert}
\newcommand{\abs}[1]{\lvert#1\rvert}
\newcommand{\Lagr}{\mathcal{L}}

\begin{document}
\section{Demonstration}
Define the Lagrangian function
\begin{equation}\label{eq:1}
    \begin{aligned}
        \Lagr = \frac{1}{2} \norm{w}^2 + C \sum_i (\xi_i + \xi_i^*) &+\sum_i \alpha_i (y_i - w\phi_i -b - \varepsilon - \xi_i) \\
        &+\sum_i \alpha_i (-y_i + w\phi_i -b - \varepsilon - \xi_i^*) \\
        &-\sum_i \mu_i\xi_i \\
        &-\sum_i \mu_i^*\xi_i^* \\
        where \quad \forall_i \: \xi_i\xi_i^* \geq 0
    \end{aligned}
\end{equation}
Variables of the two definition of the problem:
\begin{equation*}
    \begin{aligned}
            &Primal \; problem \qquad && w, \; b, \; \xi_i, \; \xi_i^* \\
            &Dual \; Problem && \alpha_i, \; \alpha_i^*, \; \mu_i, \; \mu_i^*
    \end{aligned}
\end{equation*}
Next step is try to simplify the definition of the Lagrangian wrt the problem that needs to be solved. Since the objective is to find the \textit{minimum} the developments proceeds imposing this condition.
\begin{subequations}
    \begin{align}
        &\frac {\partial \Lagr}{\partial w} = 0 \quad &&\longrightarrow \quad && w + \sum_i \alpha_i(-\phi_i) + \sum_i \alpha_i^*\phi_i = 0 \label{eq:2a}\\
        &\frac {\partial \Lagr}{\partial b} = 0 &&\longrightarrow && \sum_i -\alpha_i + \sum_i \alpha_i^* = 0 \label{eq:2b}\\
        &\frac {\partial \Lagr}{\partial \xi_i} = 0 &&\longrightarrow && C - \alpha_i - \mu_i = 0 \label{eq:2c}\\
        &\frac {\partial \Lagr}{\partial \xi_i^*} = 0 &&\longrightarrow && C - \alpha_i^* - \mu_i^* = 0 \label{eq:2d}
    \end{align}
\end{subequations}
From \eqref{eq:2a} the definition of w can be derived
\begin{equation}\label{eq:3}
    w = \sum_i (\alpha_i - \alpha_i^*)\phi_i
\end{equation}
From \eqref{eq:2b} the first constraint on the Lagrangian variables is obtained
\begin{equation}\label{eq:4}
    \sum_i (\alpha_i^* - \alpha_i) = 0
\end{equation}
While from \eqref{eq:2c}/\eqref{eq:2d} with some further development the second constraint on the Lagrangian variables can be defined
\begin{equation}
    \begin{aligned}
        &\alpha_i,\;\alpha_i^*,\;\mu_i,\;\mu_i^*\;\geq\;0 \quad \forall_i \\
        &C = \alpha_i + \mu_i\quad\longrightarrow\quad\alpha_i = C - \mu_i \\
        &\Longrightarrow\quad\alpha_i\;\in\;[0,\;C]\\
        &and\;equivalently\quad\alpha_i^*\;\in\;[0,\;C]
    \end{aligned}
\end{equation}
Simplify \eqref{eq:1} using the substitution \eqref{eq:3}
\begin{equation}
    \begin{aligned}
        \Lagr\;=\;&\frac{1}{2} \sum_i \sum_j (\alpha_i - \alpha_i^*)(\alpha_j - \alpha_j^*)\phi_i\phi_j \\
        &-\sum_i \sum_j (\alpha_i - \alpha_i^*)(\alpha_j - \alpha_j^*)\phi_i\phi_j \\
        &+\sum_i (\alpha_i - \alpha_i^*)y_i
        +\sum_i (\alpha_i - \alpha_i^*)b
        -\sum_i (\alpha_i + \alpha_i^*)\varepsilon \\
        &+\sum_i \alpha_i(-\xi_i) + \sum_i\alpha_i^*(-\xi_i^*) \\
        &-\sum_i \mu_i\xi_i - \sum_i \mu_i^*\xi_i^* \\
        & +C\sum_i \xi_i + \xi_i^*
    \end{aligned}
\end{equation}
Apply condition \eqref{eq:4} and \eqref{eq:2c} to simplify some terms and obtain the final formulation
\begin{equation}
    \begin{aligned}
        \Lagr(\alpha, \alpha^*)\;=\;&-\frac{1}{2}\sum_i\sum_j (\alpha_i - \alpha_i^*)(\alpha_j - \alpha_j^*)\phi_i\phi_j \\
        &+ \sum_i (\alpha_i - \alpha_i^*)y_i \\
        &- \sum_i (\alpha_i + \alpha_i^*)\varepsilon \\
        &With\;the\;constraints\qquad
        \begin{cases}
            \sum_i (\alpha_i^* - \alpha_i) = 0 \\
            \alpha_i\in[0,\;C] \\
            \alpha_i^*\in[0,\;C]
        \end{cases}
    \end{aligned}
\end{equation}
\end{document}

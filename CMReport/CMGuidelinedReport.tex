 \documentclass[12pt]{article}
\usepackage{graphicx} % for inserting images
\usepackage[utf8]{inputenc} % for good practice
\usepackage{amsmath} % for equations
\usepackage{amsfonts}
\usepackage{hyperref}
\usepackage[ruled,linesnumbered]{algorithm2e}
\usepackage{setspace}
\usepackage{siunitx}
\sisetup{output-exponent-marker=\ensuremath{\mathrm{e}}}

\newcommand{\norm}[1]{\left\lVert#1\right\rVert}
\newcommand{\abs}[1]{\lvert#1\rvert}
\newcommand{\Lagr}{\mathcal{L}}

\title{Support Vector Regression\\using\\Deflected Subgradient Methods}
\author{Elia Piccoli\\Nicola Gugole}
\begin{document}
	\begin{titlepage}
		\maketitle
		\pagenumbering{gobble}
	   \begin{center}
		\vspace{0.5cm}
	        \textit{A project presented for the\\Computational Mathematics for Learning and Data Analysis\\course}
	       \vfill	     
	       \includegraphics[width=0.2\textwidth]{unipi.png}\\
	       University of Pisa\\
	       Artificial Intelligence\\
			A.Y. 2020/2021\\
	            
	   \end{center}
	\end{titlepage}
	
	\newpage
	\tableofcontents
	\vspace{4cm}
	\begin{abstract}
		 Project aim is developing the implementation of a model which follows an SVR-type approach including various different kernels. The implementation uses as optimization algorithm a dual approach with appropriate choices of the constraints to be dualized, where the Lagrangian Dual is solved by an algorithm of the class of deflected subgradient methods.
	\end{abstract}
	\pagenumbering{arabic} %per rimettere i numeri di pagina
	\pagebreak

	\section{Introduction}
	SVR objective is predicting a uni-dimensional real-valued output $y$ through the use of an \textit{objective function} built by optimization using an $\varepsilon$-insensitive loss function. Another fundamental aspect about SVR is keeping the function \textit{as flat as possible} through the tuning of a $C$ parameter in order to avoid overfitting and generating a correct trade-off between accuracy and generalization.\\
	The resulting function can be generically described as:
		\begin{equation}\label{eq:1}
			f(x) = w  x + b
		\end{equation}
	Keeping the above function \textit{as flat as possible} is equivalent to an optimization problem formulated as having minimum $\norm{w}$, or, for a more convenient mathematical derivation, minimum $\norm{w}^2$, not changing the semantics of the problem.\\
	This brings us to a convex minimization problem, which will be called \textit{primal problem}:
\begin{equation}\label{eq:2}
		\min_{w,\xi_i,\xi_i^*} \frac{1}{2}\norm{w}^2+C\sum_{i}(\xi_{i} + \xi_{i}^*)
	\end{equation}
	Where $\xi$ and $\xi^*$ are called \textit{slack variables}, used in conjunction with $C$ to create a \textit{regularization factor} and consequently a \textit{penalty measure} to elements which are not part of the $\varepsilon$-tube. Slack variables allow the definition of constraints applicable to \eqref{eq:2}:
	\begin{subequations}
	\begin{align}
		&y_i - w^T\phi(x_i) - b \leq \varepsilon + \xi_i,  \label{eq:3a}\\ 
		&b + w^T\phi(x_i) - y_i \leq \varepsilon + \xi_i, \label{eq:3b}\\
		&\xi_i,\xi_i^*  \geq 0 \label{eq:3c}
	\end{align}
	\end{subequations}
	\begin{center}
		\footnotesize{$x_i$ input, $y_i$ output}
	\end{center}
	\pagebreak
	\section{Dual Representation}
	As expressed in the abstract, the implementation will follow a dual approach, which in SVR models is preferred due to the applicability and efficiency of the use of \textit{kernels}. \textit{Dual problem} formulation can be achieved defining the \textit{Lagrangian} function:
	\begin{equation}\label{eq:4}
	\begin{aligned}
		\Lagr(\alpha,\alpha^*,\mu,\mu^*) =  \ &\frac{1}{2}\norm{w}^2\\
		&+C\sum_{i=1}^{m}(\xi_{i} + \xi_{i}^*) \\
		&+ \sum_{i=1}^{m}(\alpha_i(y_i - w^T\phi(x_i) - b - \varepsilon - \xi_i))\\
		&+ \sum_{i=1}^{m}(\alpha_i^*(w^T\phi(x_i) + b - y_i - \varepsilon - \xi_i^*)) \\
		&- \sum_{i=1}^{m}(\mu_i\xi_i + \mu_i^*\xi_i^*)
	\end{aligned}
	\end{equation}
	
	From which the following optimization problem can be obtained (full derivation shown in \ref{appendixA}):
	\begin{equation}\label{eq:5}
	\begin{aligned}
		\max_{\alpha_i,\alpha_i^*} &- \frac{1}{2}\sum_i\sum_j(\alpha_i - \alpha_i^*)(\alpha_j - \alpha_j^*)K(x_i,x_j) \\
		&- \epsilon\sum_i(\alpha_i + \alpha_i^*)\\
		&+ \sum_i y_i(\alpha_i - \alpha_i^*)
	\end{aligned}
	\end{equation}
	With constraints:
	\begin{subequations}
		\begin{align}
		&\forall i \: \alpha_i,\alpha_i^* \geq 0 \qquad\qquad &&(KKT\;condition) \label{eq:6a}\\
		&\forall i \: \alpha_i,\alpha_i^* \in [0,C]  \qquad\qquad &&(from \: derivation)\label{eq:6b}\\
		&\forall i \: \sum (\alpha_i - \alpha_i^*) = 0 \qquad\qquad &&(from \: derivation)\label{eq:6c}\\
		&\forall i \: \alpha_i \alpha_i^* = 0 \qquad\qquad &&(from \: model \: construction)\label{eq:6d}
		\end{align}
	\end{subequations}
	\pagebreak


	At this point a reformulation of \eqref{eq:5} is necessary to follow the task objective, which is solving the \textit{Lagrangian Dual} maximization with a subgradient method, therefore requiring a \textit{non-differentiable function}. Such function is achievable with a simple variable substitution:
	\begin{equation*}
	    \begin{aligned}
	    &\;\beta_i \longleftarrow (\alpha_i - \alpha_i^*) \\
	    &\abs{\beta_i} \longleftarrow (\alpha_i + \alpha_i^*)
	    \end{aligned}
	\end{equation*}
	Bringing the definitive dual problem definition:
	\begin{equation}\label{eq:7}
	    \begin{aligned}
	    \max_{\beta_i} &- \frac{1}{2}\sum_i \sum_j \beta_i \beta_j K(x_i,x_j) \\
		&- \epsilon\sum_i\abs{\beta_i}\\
		&+ \sum_i y_i\beta_i\\
		&With\;the\;constraints\qquad
        \begin{cases}
            \sum_i \beta_i = 0 \\
            \;\beta_i\in[-C,\;C] \\
            \abs{\beta_i}\in[0,\;C]
        \end{cases}
        \end{aligned}
	\end{equation}
	It is important to notice how the above formulation defines a convex non-differentiable problem which still maintains the \textit{strong duality} property of SVR since it is possible to reshape \eqref{eq:5} to a quadratic problem, as shown in APPENDIX B (qui la nostra prossima idea per l'appendice B era ripartire da \eqref{eq:5} e mostrare come si può arrivare a formulare il problema quadratico tramite l'uso della matrice $Q=[K\;-K\; ; -K \; K]$, sarebbe sufficiente come dimostrazione?).\\
	Having the \textit{strong duality} property assures that the optimal solution of the dual problem coincides with the one of the primal problem. 





%APPENDIX
    \pagebreak
    \section{APPENDIX A} \label{appendixA}
Define the Lagrangian function
\begin{equation}\label{eq:APP1}
    \begin{aligned}
        \Lagr = \frac{1}{2} \norm{w}^2 + C \sum_i (\xi_i + \xi_i^*) &+\sum_i \alpha_i (y_i - w\phi_i -b - \varepsilon - \xi_i) \\
        &+\sum_i \alpha_i (-y_i + w\phi_i -b - \varepsilon - \xi_i^*) \\
        &-\sum_i \mu_i\xi_i \\
        &-\sum_i \mu_i^*\xi_i^* \\
        where \quad \forall_i \: \xi_i\xi_i^* \geq 0
    \end{aligned}
\end{equation}
Variables of the two definition of the problem:
\begin{equation*}
    \begin{aligned}
            &Primal \; problem \qquad && w, \; b, \; \xi_i, \; \xi_i^* \\
            &Dual \; Problem && \alpha_i, \; \alpha_i^*, \; \mu_i, \; \mu_i^*
    \end{aligned}
\end{equation*}
Next step is try to simplify the definition of the Lagrangian wrt the problem that needs to be solved. Since the objective is to find the \textit{minimum} the developments proceeds imposing this condition.
\begin{subequations}
    \begin{align}
        &\frac {\partial \Lagr}{\partial w} = 0 \quad &&\longrightarrow \quad && w + \sum_i \alpha_i(-\phi_i) + \sum_i \alpha_i^*\phi_i = 0 \label{eq:APP2a}\\
        &\frac {\partial \Lagr}{\partial b} = 0 &&\longrightarrow && \sum_i -\alpha_i + \sum_i \alpha_i^* = 0 \label{eq:APP2b}\\
        &\frac {\partial \Lagr}{\partial \xi_i} = 0 &&\longrightarrow && C - \alpha_i - \mu_i = 0 \label{eq:APP2c}\\
        &\frac {\partial \Lagr}{\partial \xi_i^*} = 0 &&\longrightarrow && C - \alpha_i^* - \mu_i^* = 0 \label{eq:APP2d}
    \end{align}
\end{subequations}
From \eqref{eq:APP2a} the definition of w can be derived
\begin{equation}\label{eq:APP3}
    w = \sum_i (\alpha_i - \alpha_i^*)\phi_i
\end{equation}
From \eqref{eq:APP2b} the first constraint on the Lagrangian variables is obtained
\begin{equation}\label{eq:APP4}
    \sum_i (\alpha_i^* - \alpha_i) = 0
\end{equation}
While from \eqref{eq:APP2c}/\eqref{eq:APP2d} with some further development the second constraint on the Lagrangian variables can be defined
\begin{equation*}
    \begin{aligned}
        &\alpha_i,\;\alpha_i^*,\;\mu_i,\;\mu_i^*\;\geq\;0 \quad \forall_i \\
        &C = \alpha_i + \mu_i\quad\longrightarrow\quad\alpha_i = C - \mu_i \\
        &\Longrightarrow\quad\alpha_i\;\in\;[0,\;C]\\
        &and\;equivalently\quad\alpha_i^*\;\in\;[0,\;C]
    \end{aligned}
\end{equation*}
Simplify \eqref{eq:APP1} using the substitution \eqref{eq:APP3}
\begin{equation*}
    \begin{aligned}
        \Lagr\;=\;&\frac{1}{2} \sum_i \sum_j (\alpha_i - \alpha_i^*)(\alpha_j - \alpha_j^*)\phi_i\phi_j \\
        &-\sum_i \sum_j (\alpha_i - \alpha_i^*)(\alpha_j - \alpha_j^*)\phi_i\phi_j \\
        &+\sum_i (\alpha_i - \alpha_i^*)y_i
        +\sum_i (\alpha_i - \alpha_i^*)b
        -\sum_i (\alpha_i + \alpha_i^*)\varepsilon \\
        &+\sum_i \alpha_i(-\xi_i) + \sum_i\alpha_i^*(-\xi_i^*) \\
        &-\sum_i \mu_i\xi_i - \sum_i \mu_i^*\xi_i^* \\
        & +C\sum_i \xi_i + \xi_i^*
    \end{aligned}
\end{equation*}
Apply condition \eqref{eq:APP4} and \eqref{eq:APP2c} to simplify some terms and obtain the final formulation
\begin{equation*}
    \begin{aligned}
        \Lagr(\alpha, \alpha^*)\;=\;&-\frac{1}{2}\sum_i\sum_j (\alpha_i - \alpha_i^*)(\alpha_j - \alpha_j^*)\phi_i\phi_j \\
        &+ \sum_i (\alpha_i - \alpha_i^*)y_i \\
        &- \sum_i (\alpha_i + \alpha_i^*)\varepsilon \\
        &With\;the\;constraints\qquad
        \begin{cases}
            \sum_i (\alpha_i^* - \alpha_i) = 0 \\
            \alpha_i\in[0,\;C] \\
            \alpha_i^*\in[0,\;C]
        \end{cases}
    \end{aligned}
\end{equation*}

\end{document}